%% This is an example first chapter.  You should put chapter/appendix that you
%% write into a separate file, and add a line \include{yourfilename} to
%% main.tex, where `yourfilename.tex' is the name of the chapter/appendix file.
%% You can process specific files by typing their names in at the 
%% \files=
%% prompt when you run the file main.tex through LaTeX. 
% MBA, CBA
\chapter{Method-\rom{2}: Movement-based Approach on Constraint for WIMP}
After exploring the first limits on WIMPs with Method-\rom{1}, the systematic error should be explored with Method-\rom{2}, considering the almost-authentic system. In this analysis, two cases are taken into account. The first one is the straight-line case, which is actually similar to our Method-\rom{1} but this time the 

\section{Introduction to the method}

Instead of the implantation of the interaction counts as being applied in the previous section, the movement-based approach, which suggests the step-by-step movement-and-interaction action, is revealing the genuine physical process. It shifts with a tiny length at a time, and rolls a dice to see if it would interact with the atom or not, in accordance with the Possion distribution. The method bewritted in this section is probed as the way to check the systematic difference compared with the count-based approach.\\

\begin{figure}
\begin{center}
\begin{tikzpicture}[node distance=2cm]
\node (start) [startstop, align=center] {Input: A WIMP($M_{\chi}$,$E_{I}=\frac{1}{2}M_{\chi}V_{I}^{2}$)};
\node (Length) [startstop, below of=start, align=center] {Move a distance given by the formula\ref{Interaction count modification}.  };

\node (Dice) [startstop, below of=Length, align=center] {Roll a dice!};

\node (Yes) [startstop, below of=Dice, align=center, xshift=-2cm] {Yes};
\node (No) [startstop, below of=Dice, align=center, xshift=2cm] {No};

\node (AnInteraction) [startstop, below of=Yes, align=center] {$\bullet$Run an interaction \\ A. With the scattering angle \\ B. Without the scattering angle};

\node (endbf) [startstop, below of=AnInteraction, align=center, xshift=2cm, yshift=-2cm] {(The energy of WIMP < $E_{th}$])? \\ or \\ (Reach the detector)? };

\node (rightofendbf) [startstop, right of=endbf, align=center, xshift=5cm]{Keep going on!};

\node (end) [startstop, below of=endbf, align=center, yshift=-0.5cm] {Output: A WIMP($M_{\chi}$,$E_{F}=\frac{1}{2}M_{\chi}V_{F}^{2}$)};

\node (Bang) [startstop, below of=end, align=center, xshift=-2cm] {Bang!};
\node (Missing) [startstop, below of=end, align=center, xshift=2cm] {Missing!};

\draw [o-o] (start) -- node{Start from the outmost layer of air}(in1);
\draw [arrow] (in1) -- node{}(Loop);

\draw [arrow] (Dice) -- node{1}(Yes);
\draw [arrow] (Dice) -- node{0}(No);
\draw [arrow] (Yes) -- node{}(AnInteraction);
\draw [arrow] (AnInteraction) -- node{Check}(endbf);
\draw [arrow] (No) -- node[yshift=-1.2cm]{Check}(endbf);

\draw [arrow] (endbf) -- node{No}(rightofendbf);
\draw [arrow] (rightofendbf) |- node{}(Length);
\draw [arrow] (endbf) -- node{Yes}(end);
\draw [arrow] (end) -- node{$E_{\chi}$ {\bf>} $E_{th}$}(Bang);
\draw [arrow] (end) -- node{$E_{\chi}$ {\bf<}  $E_{th}$}(Missing);

\end{tikzpicture}
\end{center}
\caption{The process of the movement-based approach.} \label{Interaction_1}
\end{figure}

\begin{enumerate}
	\item Generate a WIMP with a velocity based on the dark matter halo model and a random direction.\\
	\item With the formula\ref{Interaction count modification}, which is the same as \ref{Interaction count} but a slight movement, we can get the length that a WIMP goes through for the next step:\\
	\begin{equation}
        \label{Interaction count modification}
	L = \frac{N}{D \times \sigma}
        \end{equation}
        $L$ = The length a WIMP passes through the material\\
        $D$ = The density of the material that a WIMP passes through\\
        $\sigma$ = The total cross-section\\
        $N$ = The interaction count\\ 
	\item Since the interaction over two times for an interaction is impossible, a dice should be rolled with the Possion distribution which the $\lambda$ is 0.001 to see whether it will interact with the atom or nor. After every interaction, there are two cases:\\
	\begin{enumerate}
		\item Without the scattering angle:\\
		The WIMP continues running straightforward without any consideration.\\
		\item With the scattering angle:\\
		Derived from the two-dimensional collision, the following formula can be found in \ref{}:
		\begin{equation}
                 \label{Interaction count modification}
    		tan(\theta_{lab}) = \frac{sin(\theta_{rest})}{cos(\theta_{rest})+\frac{M1}{M2}}
                 \end{equation}
		M1 is the mass of the WIMP, M2 is the mass of the atom. $\theta$ is randomly chosen between 0 to 180 degree. After the final velocity of the WIMP as well as the scattering angle are determined, the direction of it can be decided as well.\\
	\end{enumerate} 
	\item Loop between 2. to 4. till the WIMP reaches the detector or the energy of it is smaller than the threshold.\\
\end{enumerate}

\section{Two-case movement}
To compare between these two methods for estimating the systematic error for method1, the sensitivity lines originating from these two methods would be overlapped to see the difference. In the following sections, two cases, including the straight trajectory as well as the bending trajectory, are competed to see.
\subsection{S-mode case}
The same as the count-based movement, the case here is using S-mode as the assumption to be the trajectory for WIMPs. In this case, the first thing that should be answered is: With the intuitive way of step-by-step movement, what happens to our final results on setting limits for WIMPs? Is there a significant difference between these two methods?\\

In Fig.\ref{}, the plot shows the pretty clear pattern that under this mode with the movement-based method, there is no significant difference from the count-based method. From the theoretical point, actually the expectation value is very close to the count-based values, so the similar results are envisioned.

\begin{figure}[]
\captionsetup[subfigure]{}
\centering

\subcaptionbox{(a)}{\includegraphics[width=0.4\columnwidth]{/Users/yehchihhsiang/Desktop/Analysis/CDEX_Analysis_method/Codes/Thesis_Plot/Sensitivity_Line/All_0P2GeV_STS_TEXONO_BR.pdf}}
\subcaptionbox{(b)}{\includegraphics[width=0.4\columnwidth]{/Users/yehchihhsiang/Desktop/Analysis/CDEX_Analysis_method/Codes/Thesis_Plot/Sensitivity_Line/All_20GeV_STS_TEXONO_NU.pdf}} 

\caption{}\label{}
\end{figure}

\subsection{B-mode case}
In finale, as the WIMPs pass through the material, they could be scattered with the atoms and derail to other directions, and possibly, some of them could pass longer length, along with accompanying more interaction counts before they arrive at the detector. Even more, some of them could be bent off the original direction extremely that they can't reach the detector. In this study, the bending case having the impact on setting limits for WIMPs is explored.\\ 

In this case, there is a slight difference compared with the previous studies. In order to make the studies more simpler, the CRESST case, which is one of the surface experiments, is simulated for the systematic error. Being distinct from the S-mode, which forces WIMPs to fly to the detector directly with a straight line, the method here is to let it freely move with the scattering processes till they attain the surface of the earth.\\ 

\begin{equation}
\label{Interaction count modification}
L_{B/S} = \frac{ \text{The B-mode path length}}{\text{The S-mode path length}} = \frac{L_{B}}{L_{S}}
\end{equation}

In Fig.\ref{}, the comparison between two cases, including S-mode and B-mode, is shown. Overall, the goal from this study is to find out the biggest systematic error and mark them for our study, so the thing that should be done here is to find out the biggest difference between these two sensitivity lines. As the studies done in the previous section, the boundaries for setting limits mostly emerge in the stage2(lower boundary) and stage3(upper boundary), the representative lines for these two stages from two modes are depicted to define our systematic error. Comparing between these two modes, some of the intriguing issue can be made as follows:

\begin{enumerate}
\item  Stage2:\\
There is a clear tendency that can be observed is that, when the mass becomes smaller and smaller, the red line, meaning the B-mode case, is getting lefter, closing to the vacuum-constrained line. The proof of this is in Fig.\ref{}, it shows the ratio of the B-mode path length to the S-mode path length. When the mass of WIMP is getting lower, it could be bent dramatically when it gets to the earth, and quickly return back to the surface, leading to the smaller ratio as seen in figure. The smaller the ratio, the higher the energy that would be preserved by WIMPs, resulting in the near vacuum-like case. 
\item  Stage3:\\
The opposite case happens to the stage3, compared with the stage2. As the mass grows lower, the WIMPs should pass more longer case to arrive at the detector, furthermore, some of them could be off the road to the surface. As a result, when the mass is getting lower, the red line becomes more lefter. The Fig\ref{} proves our point of view that the higher mass corresponds to the higher $L_{B/S}$, 
\end{enumerate}

\begin{figure}[]
\captionsetup[subfigure]{}
\centering

\subcaptionbox{(a)}{\includegraphics[width=0.4\columnwidth]{/Users/yehchihhsiang/Desktop/Analysis/CDEX_Analysis_method/Codes/Thesis_Plot/Sensitivity_Line/All_20GeV_STS_Bent.pdf}}
\subcaptionbox{(b)}{\includegraphics[width=0.4\columnwidth]{/Users/yehchihhsiang/Desktop/Analysis/CDEX_Analysis_method/Codes/Thesis_Plot/Sensitivity_Line/All_2GeV_STS_Bent.pdf}} 
\subcaptionbox{(b)}{\includegraphics[width=0.4\columnwidth]{/Users/yehchihhsiang/Desktop/Analysis/CDEX_Analysis_method/Codes/Thesis_Plot/Sensitivity_Line/All_0P2GeV_STS_Bent.pdf}} 

\caption{}\label{}
\end{figure}



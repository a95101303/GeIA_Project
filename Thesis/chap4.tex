%% This is an example first chapter.  You should put chapter/appendix that you
%% write into a separate file, and add a line \include{yourfilename} to
%% main.tex, where `yourfilename.tex' is the name of the chapter/appendix file.
%% You can process specific files by typing their names in at the 
%% \files=
%% prompt when you run the file main.tex through LaTeX.

\chapter{Analysis Strategies}
%%%%%%%%%%%%%%%%%%%%%%%%%%%%%%%%%%%%%%%%%%%%%%%%%%%%%%%%%%%%%%%%%%%%%%%%%%%%%%%%%
\section{Introduction}
In our studies, the major aim is to figure out that after a bunch of WIMPs come over to our detector through various layers of solid earth, air and shielding with the disparate densities, which are claimed by our earth, air model as well as the structure of the shielding surrounding by the germanium detector in the previous sections, how many WIMPs can still survive and retain energized to give our detector "multiple hits" after lots of "barricades" stationing on their way to the detector.\\

In the first place, since some of the terminologies are applied throughout our studies frequently, in the first section, two subsets of definitions, including vacuum/ earth effect cases, as well as the modes of WIMPs flying to our detector, would be well-written. Then, as the crux process of all fashions, which are detailed in the inferior sections, "an interaction" between the atom and the WIMP should be equipped in advance. Later on, three interacting channels, which are employed to make the measurement on WIMPs realized, are depicted as the highlight of our studies in the following section. In the end, speaking of setting limits on WIMPs, the exclusion plot, which is the battleground for all direct-detection experiments on detecting WIMPs, must be conceptualized ahead of time, then the vacuum-constrained cases are delivered as the specifications for the preparation of taking the earth effect into account.

%%%%%%%%%%%%%%%%%%%%%%%%%%%%%%%%%%%%%%%%%%%%%%%%%%%%%%%%%%%%%%%%%%%%%%%%%%%%%%%%%
\section{Definition Clarifications}
\subsection{Vacuum Case\&Earth Effect Case}
There are two types of cases that should be defined as the consensus for our studies:
\begin{enumerate}
	\item Vacuum case:\\
	 Vacuum, which means "devoid of matter in the space", is an analogy in our studies, indicates that the intensity of interaction between WIMPs and atoms is too low, resulting in all material on the route WIMPs traverse to the detector seems transparent to WIMPs. In other words, any material can't make them decelerated supposedly. The interactions with the material are { \bf totally overlooked}, except their interactions with our detector.
	\item Earth Effect Case:\\
	In contrast to the vacuum case, the interactions between WIMPs and atoms, comprising the material of solid earth, air, and shielding, are taken into account precisely. 
\end{enumerate}

\subsection{Trajectory for WIMPs: S-mode\&B-mode}
In our research, two modes for simulating the flying WIMPs from our universe are as follows:
\begin{enumerate}
	\item S-mode:\\
	The trajectory of the S-mode is a straight line. Under this mode, all of the WIMPs in our universe fly directly to our detector with a straight line defaultly. The scheme for this mode is in Fig.\ref{}.
	\item B-mode:\\
	The trajectory of the B-mode is a bending line. Under this mode, WIMPs could be bent by the atoms when they come to our detector. The chart for this mode is in Fig.\ref{}.
\end{enumerate}


\section{Core: The loop for An Interaction}
Since the earth effect should be quantified by premeditating the effect of deceleration on the moving WIMPs, the basic interacting process should be demonstrated.\\

In Figure \ref{Interaction}, the process of an interaction between a WIMP and an atom is performed. Having known velocity and mass given at the start for WIMP, the energy loss can be obtained randomly based on the spectrum of $\frac{d\sigma}{dT}$, portraying which energy loss should be opted, according to the cross-section, which is referred to the possibility. In the end, after the energy loss is taken into account, the final velocity and energy of WIMP can be gained.\\

All of the means made use of in our research stick with "an interaction", which is the basic unit erected as the kernel of the mediums since WIMPs can transport an amount of energy to atoms of "solid earth, air, and shielding" through numerous interactions, and this unit plays an important role throughout the entire studies. \\

\begin{figure}
\begin{center}
\begin{tikzpicture}[node distance=2cm]
\centering
\node (start) [startstop, align=center] {1.Input: A WIMP($M_{\chi}$,$E_{I}=\frac{1}{2}M_{\chi}V_{I}^{2}$)};
\node (in1) [io, below of=start, align=center] {2.$\frac{d\sigma}{dT}$($M_{\chi}$,$V_{I}$,$\sigma$,$A$) $\rightarrow$ $E_{Loss}$ \\ $E_{F}$ = $E_{I}$ - $E_{Loss}$  };
\node (end) [startstop, below of=in1, align=center] {3.Output: A WIMP($M_{\chi}$,$E_{F}=\frac{1}{2}M_{\chi}V_{F}^{2}$)};


\draw [arrow] (start) -- node{given $\sigma$, $A$}(in1);
\draw [arrow] (in1) -- node{}(end);
\end{tikzpicture}
\end{center}
\caption{The process of an interaction for a WIMP with an atom.} \label{Interaction}
\end{figure}


\section{Three Types of Interacting channels in the Detector}
Because the various channels of interaction are sensitive to the different $M_{\chi}$ with the different cross-sections, comprehending the mechanisms of those channels, and authentically figuring out the rates of WIMPs by those should be well-found. In this section, to begin with, the WIMP-nucleus elastic scattering($\chi-N$ Scattering), which is the typical mechanism for detecting WIMPs, would be reviewed with the fundamental deduction for it. Based on the $\chi-N$ Scattering, two genres of the signal originating from electrons and photons by mechanisms of Migdal Effect and Bremsstrahlung individually were proposed to probe the low-mass WIMPs by some of the theoretical research recently. Therefore, Both of the mechanisms are also taken into consideration as two channels to probe WIMPs, and will be reviewed in this section as well. Hopefully, with the help of these two channels, the low-mass WIMPs with the unexplored cross-sections can be unveiled.

\subsection{$\chi-N$ Scattering}
Traditionally, the atomic electrons around the nucleus of the target material is assumed to have the same motion as the recoil nucleus, meaning that electrons and recoil nucleus are considered as a whole to interact with WIMPs. After the whole Ge is scattered by WIMPs, an amount of energy from WIMPs will be deposited in it. In order to formulate the differential event rate of $\chi-N$ Scattering for the signal of spin-independent WIMPs, the following deduction is utilized in our analysis.

In the first place, the differential WIMP-nucleon cross section can be obtained by the following formula:
\begin{equation}
\frac{d\sigma_{WN}(q)}{dq^{2}} = \frac{\sigma_{0WN}F^{2}(q)}{4\mu_{A}^{2}v^{2}}
\end{equation}
where $\sigma_{0WN}$ is the zero-momentum WIMP-nucleon cross section, $v$ is the velocity of the WIMP in the lab frame, $\mu_{A} \equiv \frac{M_{A}M_{\chi}}{M_{A}+M_{\chi}}$ is the reduced mass of the WIMP-nucleus system, $q$ is the momentum transfer, and $F$ is the nuclear form factor, depending on the model. The Woods-Saxon form factor is suggested to be the good approximation for the spin-independent WIMP, which is suitable for our case:
\begin{equation}
F(q) = \frac{3[\text{sin}(qr_{n})-qr_{n}\text{cos}(qr_{n})]}{(qr_{n})^{3}}e^{\frac{-(qs)^{2}}{2}}
\end{equation}
where 
\begin{equation}
r_{n}^{2} = (1.23A^{1/3}-0.60 \text{fm})^2 + \frac{7}{3}(0.52\pi \text{fm})^{2}-5s^{2}
\end{equation}
$s=0.9$fm, $A$ is the atomic mass of the target, and 
\begin{equation}
q=\sqrt{2M_{A}E_{R}}
\end{equation}
Here, $M_{A}$ is the mass of the target, and $E_{R}$ is the recoil energy.
As the material of the target applied in the different experiments is varying, the dependence of $\sigma_{0WN}$ on the material of the target must be factored out with the new form:
\begin{equation}
\sigma_{0WN,SI} = \sigma_{SI} \frac{\mu_{A}^{2}}{\mu_{n}^{2}} A^{2}
\end{equation}
where
\begin{equation}
\mu_{n} \equiv \frac{M_{n}M_{\chi}}{M_{n}+M_{\chi}}$
\end{equation}

When the differential WIMP-nucleon cross section is built up, the differential cross-section, which will be used in calculating the differential event rate, can be figured out:
\begin{equation}
\frac{d\sigma_{WN}(q)}{dq^{2}} \times 2A = \frac{d\sigma_{WN}(q)}{dq} = \frac{d\sigma_{WN}(q)}{dT} = \frac{d\sigma}{dE_{R}}
\end{equation}

In the end, the differential event rate can be calculated as follows:
\begin{equation}
\frac{dR}{dE_{R}} = N_{T} \frac{\rho_{\chi}}{m_{\chi}}} \int d^{3}\vec{v}vf_{v}(\vec{v}+\vec{v_{E}})\frac{d\sigma}{dE_{R}}
\end{equation}

\subsection{Migdal Effect}
In reality, since the electrons surrounding by the nucleus should take some time to catch up the motion of the nucleus, leading to the effects of ionization and excitation, the signal of the electrons, which can't be generated by the $\chi-N$ scattering, can be acquired for exploring the lower $M_{\chi}$. Being similar to the one adopted in the $\chi-N$ scattering, the differential event rate for the Migdal effect can be expressed as follows:
\begin{equation}
\frac{d^{2}R}{dE_{EM}E_{R}} = N_{T} \frac{\rho_{\chi}}{m_{\chi}}} \int d^{3}\vec{v}vf_{v}(\vec{v}+\vec{v_{E}})\frac{d^{2}\sigma}{dE_{EM}dE_{R}}
\end{equation}
where
\begin{equation}
\frac{d^{2}\sigma}{dE_{EM}E_{R}} = \frac{d\sigma}{dE_{R}} \frac{1}{2\pi}\Sigma_{n,l}\frac{d}{dE_{EM}}p^{c}_{qe}(nl \rightarrow (E_{EM} - E_{nl} ))}
\end{equation}

\subsection{Bremsstrahlung}

$\frac{d^{2}\sigma}{dE_{\gamma}E_{R}} = \frac{4\alpha}{3\pi E_{\gamma}} \frac{E_{R}}{m_{N}c^{2}} |f(E_{\gamma}) |^{2} \times \frac{d\sigma}{dE_{R}} \Theta (E_{\gamma,max}-E_{\gamma})

$f(E_{\gamma}) = f_{1}(E_{\gamma})+if_{2}(E_{\gamma})$

$\frac{d^{2}R}{dE_{\gamma}E_{R}} = N_{T} \frac{\rho_{\chi}}{m_{\chi}}} \int d^{3}\vec{v}vf_{v}(\vec{v}+\vec{v_{E}})\frac{d^{2}\sigma}{dE_{\gamma}dE_{R}}$

%%%%%%%%%%%%%%%%%%%%%%%%%%%%%%%%%%%%%%%%%%%%%%%%%%%%%%%%%%%%%%%%%%%%%%%%%%%%%%%%%
\section{Battleground for WIMPs: Exclusion Plot}
The territory on the map of $\sigma_{SI}$ and $M_{\chi}$, which is also named as "exclusion plot", is adopted to compare the capabilities between a variety of the experiments. The more territory the experiment can advocate, the higher the possibility that it can discover the WIMP, on the other hand, if there is no excess signal from the experiment, it can exclude more regions and encourages other developing experiments to elongate their aptness toward proclaiming more unknown territory. In Fig.\ref{}, the territories claimed by other experiments before are demonstrated. There are three pieces of information that can be acquired from each enclosing region:\\
\begin{enumerate}
\item  Upper boundary: Associated with the location of the detector incorporating with the surrounding shielding.  
\item  Left boundary: The threshold of the detector, relying on the interacting channel considered at the moment.
\item  Lower boundary: Related to the "data-taking period $\times$ the mass of the detector".
\end{enumerate}
From this plot, another intriguing phenomenon is that, if the same territory is asserted by two or more experiments, and one of them assures there is no excess signal popping up from their observation, then it's a cross-check on the findings for each other. Because sometimes, one of them could find out the excess signal within the region which is already excluded by other experiments, then there could be something that is not pondered in the studies, leading to the fake signal, such as electronic noise, the dust on the detector, and so on.\\

In the following studies, since the capabilities of our experiments including TEXONO and CDEX should antagonize with other experiments, the same plot would be utilized as the constraint on WIMPs is accomplished with the certain way under the different conditions.\\

In parallel, because of the gargantuan room for the low-mass region to be investigated with the more lower threshold of the detector, the studies on comprehending the fundamental solid-state physics emerging in the crystal with regard to a stack of parameters, including temperature, impurity level, etc. are terminated. Along with that, the method of amplifying the signal of WIMP directly is also probed for opening another possibility for the discovery of the WIMP. The extension of the studies would be summarized in the section of the "extension studies". 

\section{Vacuum-constrained limits on WIMPs}
Before hunting for the constraint on WIMPs with the earth effect, the vacuum case, meaning that there is no earth effect on WIMPs, should be explored as our criteria, and the numbers of cross-sections captured from this section are considered to be the most considerable preparatory.\\  
\subsection{Concept of Setting limits with Data}
As the model and the data involve divergent information:\\
\begin{enumerate}
\item  Data: "Unknown Physics" + "Known Physics"
\item  Model: "Unknown Physics"
\end{enumerate}
The philosophy of setting limits on WIMPs is that, since the data encompasses more information theoretically, the event number from the assumed models, including those three channels delineated before, can't surpass the data. In our studies, as the form of the event is "rate of WIMP", the spectrum of the rates from the theoretical predictions for WIMPs can't surpass the spectrum obtained from our data. On the other hand, if the rates of the WIMPs from those channels are higher than the ones gotten from the data, the models should be excluded for the unreasonable expectation.\\

Because of the higher rate of WIMP from any of the interacting channel in the detector with respect to the higher cross-section, any model of WIMP can initiate being precluded with the critical cross-section that causes the certain point of modeling spectrum to intersect with the certain upper bound of the point, which corresponds to the 90\% confidence level, derived from the binned Possion method, in the data.

\subsection{Exemplars for cases}
 In Figure\ref{Vacuum-constrained WIMPs}, the data is presented with two Migdal-effect spectrums at masses equal to 1GeV as well as 0.05GeV. As a result of the unearthing of the critical cross-sections, which correspond to the 90\% upper limits confidence level as listed in the figure, the limits set by vacuum are concluded for both masses. When the cross-sections are higher than the critical ones, the spectrums of models will outpace the measured one, which should be entirely excluded as they contradict the core tenet composed before. \\ 
    \begin{figure}[h]
    \includegraphics[scale=0.5]{/Users/yehchihhsiang/Desktop/Analysis/CDEX_Analysis_method/Codes/Thesis_Plot/Lower_Limit_Plot.png}
    \centering
    \caption{The measured spectrum for our analysis (black point), with L/M-shell x-ray contributions from the cosmogenic nuclides in the germanium crystal subtracted. The bin width is 50eVee, and the energy range is 0.16–2.16keVee. The blue dash-dotted line and red dash line are the expected χ-N spectra due to Migdal effect at mχ equal to 50  and 1.0  , at cross-section corresponding to the upper limit at 90\% confidence level, derived by the binned Poisson method.} \label{Vacuum-constrained WIMPs}
    \end{figure}
  
After all of the critical cross-sections are sought for all masses under the interacting channels we are investigating, the numbers can be applied in being the datums for the vacuum cases. In the next chapter, the earth effect case, considering the interaction between the material and the WIMPs, inducing the energy loss of WIMPs when they go forward toward the detector, is probed.\\

\subsection{Exclusion plot for WIMPs}
Basically, 
%%%%%%%%%%%%%%%%%%%%%%%%%%%%%%%%%%%%%%%%%%%%%%%%%%%%%%%%%%%%%%%%%%%%%%%%%%%%%%%%%

\chapter{Method-\rom{1}: Count-based Approach on Constraint for WIMP}


\section{Introduction to the method}
In Figure \ref{Interaction_1}, it shows the count-based approach, and the process is as follows:\\

\begin{figure}
\begin{center}
\begin{tikzpicture}[node distance=2cm]
\node (start) [startstop, align=center] {1. Input: A WIMP($M_{\chi}$,$E_{I}=\frac{1}{2}M_{\chi}V_{I}^{2}$)};
\node (in1) [io, below of=start, align=center] {2. $N_{Air}$(\ref{Earth_Interaction}), $N_{Earth}$(\ref{Air_Interaction}), $N_{Shielding}$(\ref{Shielding_Interaction}) can be figured out};
\node (Loop) [startstop, below of=in1, align=center] {3. $\bullet$ Run an interaction };
\node (endbf) [startstop, below of=Loop, align=center, yshift=-0.5cm] {($E_{\chi}$ < $E_{th}$)? \\ or \\ Deplete all interaction counts};

\node (endbf1) [startstop, right of= endbf, align=center, xshift=5cm] {$N_{i} = N_{i} - 1$ \\  i=Air, Earth or shielding};
\node (end) [startstop, below of=endbf, align=center, yshift=-0.5cm] {4. Output: A WIMP($M_{\chi}$,$E_{F}=\frac{1}{2}M_{\chi}V_{F}^{2}$)};

\node (Bang) [startstop, below of=end, align=center, xshift=-2cm] {Bang!};
\node (Missing) [startstop, below of=end, align=center, xshift=2cm] {Missing!};



\draw [arrow] (start) -- node{given direction}(in1);
\draw [o-o] (in1) -- node{Initiate the loop of all interactions! \\ Order: $N_{Air}$ $\rightarrow$ $N_{Earth}$ $\rightarrow$ $N_{Shielding}$}(Loop);
\draw [arrow] (Loop) -- node{Check}(endbf);
\draw [arrow] (endbf) -- node{No}(endbf1);
\draw [arrow] (endbf1) |- node{Interact again!}(Loop);
\draw [arrow] (endbf) -- node{Yes}(end);

\draw [arrow] (end) -- node{$E_{\chi}$ {\bf>} $E_{th}$}(Bang);
\draw [arrow] (end) -- node{$E_{\chi}$ {\bf<}  $E_{th}$}(Missing);

\end{tikzpicture}
\end{center}
\caption{The process of the count-based approach.} \label{Interaction_1}
\end{figure}

%%%%%%%%%%
\begin{comment}
\node (dec1) [decision, below of=pro1, yshift=-0.5cm] {Decision 1};
\node (pro2a) [process, below of=dec1, yshift=-0.5cm] {Process 2a};
\node (pro2b) [process, right of=dec1, xshift=2cm] {Process 2b};
\node (out1) [io, below of=pro2a] {Output};
\node (stop) [startstop, below of=out1] {Stop};
\end{comment}
\begin{comment}
\draw [arrow] (in1) -- (pro1);
\draw [arrow] (pro1) -- (dec1);
\draw [arrow] (dec1) -- node[anchor=east] {yes} (pro2a);
\draw [arrow] (dec1) -- node[anchor=south] {no} (pro2b);
\draw [arrow] (pro2b) |- (pro1);
\draw [arrow] (pro2a) -- (out1);
\draw [arrow] (out1) -- (stop);
\end{comment}
%%%%%%%%%%


\begin{enumerate}
	\item Generate a WIMP with a velocity based on the dark matter halo model and a random direction.\\
	\item Let it flee through the material of "solid earth, air and shielding" along a straight from our universe to the detector. Given the direction of the WIMP, the interaction counts in the different layers of three components can be estimated as follows:\\
	\begin{equation}
    \label{Interaction count}
    L \times D \times \sigma = N
    \end{equation}
    $L$ = The length a WIMP passes through the material\\
    $D$ = The density of the material that a WIMP passes through\\
    $\sigma$ = The total cross-section\\
    $N$ = The interaction count\\ 
	Since a variety of the densities in the different layers of solid earth, air and shielding should be regarded, the formulae for the aggregation of all compartments as follows:\\
	\begin{equation}
    \label{Earth_Interaction}
	\mathop{\sum_{i=1}^{18}} L_i \times D_i \times \sigma = N_{Air}
    \end{equation}
	\begin{equation}
    \label{Air_Interaction}
	\mathop{\sum_{i=1}^{18}} L_i \times D_i \times \sigma = N_{Earth}
    \end{equation}
	\begin{equation}
    \label{Shielding_Interaction}
	\mathop{\sum_{i=1}^{18}} L_i \times D_i \times \sigma = N_{Shielding}
    \end{equation}

	\item After reckoning the total counts for three components, the loop with the number of interactions can be launched. Run an interaction at a time. 
	  
	\item While the total count is depleted, or the energy of WIMP is smaller than the $E_{th}$, the velocity will be recorded soon and the process would be shut down.\\
\end{enumerate}

From this fashion, the merit of economizing time can be obtained as the count is the only parameter we should acquire, and let it enforce the loop of interactions directly till the energy of it is smaller than the threshold or attaining the detector. 

\section{Three-stage blockage for earth effect}
\subsection{Introduction to each stage}
Basically, the three-stage blocking effect can be foreseen as three components on the way WIMPs fly to our detector. All WIMPs could be blocked from reaching the detector at a certain stage, depending on the interaction counts.\\
\begin{enumerate}
\item Stage-1: No-effect stage\\
$\rightarrow$When the cross-section is too small to make a WIMP have an interaction with the atom, any material won't have an impact on the WIMPs, and they can freely pass through three components. 
\item Stage-2: Solid-earth-effect stage\\ 
$\rightarrow$Since the high densities of the solid earth with the long lengths that WIMPs pass by, the solid earth is the priority of involving interacting with the WIMPs. 
\item Stage-3: Shielding-and-air effect stage\\
$\rightarrow$At last, the shielding of the detector amalgamating with the air would entirely block all of the WIMPs from getting to the detector. 

\end{enumerate}

For the case of the TEXONO experiment, since it is being manipulated on the surface without the weighty shielding and heavy building nearby, most of the WIMPs could be still alive till the third stage. In another case, if you choose the extreme case, such as CDEX considered in our studies, all of the WIMPs could be blocked when transiting to the second stage as there is an extremely gigantic mountain covering its detector. By and large, It depends on the place where you set up for your detector and the structure of the environment, resulting in the disparage outcome for the upper limits of the cross-sections from a variety of the experiments located at the different positions. \\ 

\subsection{Velocity distributions}
Ahead of recognizing the spectrums for three interacting channels, which are used to set up limits for WIMPs, the velocity distributions for the different masses at the different cross-sections should be made out in the first place.\\ 

In Figure\ref{}, the velocity distributions with the gradually heightened cross-section are demonstrated with the count-based method at $M_{\chi}=10GeV$. The phenomenon, which is that the fewer and fewer event can be observed with the detectable energies with respect to the enhanced cross-sections, can be espied with the red lines in the figure.\\ 

The cardinal concept is that when the cross-section is getting larger, the interacting strength between the WIMP and the atom is getting stronger, implying the more energy transfusion of WIMP for every interaction with the material of solid earth, air, and shielding. After all, the energies of all WIMPs are weaker than the detector threshold, leading to the actuality that none of them can be detected by our detector. \\

\begin{figure}[]
\captionsetup[subfigure]{}
\centering

\subcaptionbox{(a)}{\includegraphics[width=0.4\columnwidth]{/Users/yehchihhsiang/Desktop/Analysis/CDEX_Analysis_method/Codes/Thesis_Plot/The_Basic_Flux/1_Used.pdf}}
\subcaptionbox{(b)}{\includegraphics[width=0.4\columnwidth]{/Users/yehchihhsiang/Desktop/Analysis/CDEX_Analysis_method/Codes/Thesis_Plot/The_Basic_Flux/1_Used.pdf}} 
\subcaptionbox{(a)}{\includegraphics[width=0.4\columnwidth]{/Users/yehchihhsiang/Desktop/Analysis/CDEX_Analysis_method/Codes/Thesis_Plot/The_Basic_Flux/1_Used.pdf}}
\subcaptionbox{(b)}{\includegraphics[width=0.4\columnwidth]{/Users/yehchihhsiang/Desktop/Analysis/CDEX_Analysis_method/Codes/Thesis_Plot/The_Basic_Flux/1_Used.pdf}} 
\subcaptionbox{(a)}{\includegraphics[width=0.4\columnwidth]{/Users/yehchihhsiang/Desktop/Analysis/CDEX_Analysis_method/Codes/Thesis_Plot/The_Basic_Flux/1_Used.pdf}}
\subcaptionbox{(b)}{\includegraphics[width=0.4\columnwidth]{/Users/yehchihhsiang/Desktop/Analysis/CDEX_Analysis_method/Codes/Thesis_Plot/The_Basic_Flux/1_Used.pdf}} 

\caption{The distributions of the velocity at 10GeV under the different cross-sections are shown. }\label{}
\end{figure}

\subsection{Recoil Spectrums}
In Figure\ref{}, Figure\ref{} and Figure\ref{}, they exhibit a series of the plots mirroring the
spectrums due to the Migdal Effect with regard to the cross-sections from the
the smaller one to the bigger one under the different stages. 
In Figure\ref{}, the no-effect stage is performed clearly with the evidence that two lines are overlapping together. 
In Figure\ref{}, the spectrums of the earth cases sluggishly lessen as the solid-earth-effect stage commence involving in. In the end, in Figure\ref{}, the spectrums of the earth effect cases dramatically drop when the cross-section is magnified, resulting in all of the WIMPs are blocked by the shielding and air, which is called the "shielding-and-air effect stage".

\begin{figure}[]
\captionsetup[subfigure]{}
\centering

\subcaptionbox{(a)}{\includegraphics[width=0.4\columnwidth]{/Users/yehchihhsiang/Desktop/Analysis/CDEX_Analysis_method/Codes/Thesis_Plot/Recoil_Spectrum/0P2GeV_MD_BR_1.pdf}}
\subcaptionbox{(b)}{\includegraphics[width=0.4\columnwidth]{/Users/yehchihhsiang/Desktop/Analysis/CDEX_Analysis_method/Codes/Thesis_Plot/Recoil_Spectrum/0P2GeV_MD_BR_2.pdf}} 
\subcaptionbox{(b)}{\includegraphics[width=0.4\columnwidth]{/Users/yehchihhsiang/Desktop/Analysis/CDEX_Analysis_method/Codes/Thesis_Plot/Recoil_Spectrum/0P2GeV_MD_BR_3.pdf}} 
\subcaptionbox{(b)}{\includegraphics[width=0.4\columnwidth]{/Users/yehchihhsiang/Desktop/Analysis/CDEX_Analysis_method/Codes/Thesis_Plot/Recoil_Spectrum/0P2GeV_MD_BR_4.pdf}} 
\subcaptionbox{(b)}{\includegraphics[width=0.4\columnwidth]{/Users/yehchihhsiang/Desktop/Analysis/CDEX_Analysis_method/Codes/Thesis_Plot/Recoil_Spectrum/0P2GeV_MD_BR_4.pdf}} 
\subcaptionbox{(b)}{\includegraphics[width=0.4\columnwidth]{/Users/yehchihhsiang/Desktop/Analysis/CDEX_Analysis_method/Codes/Thesis_Plot/Recoil_Spectrum/0P2GeV_MD_BR_4.pdf}} 

\caption{}\label{}
\end{figure}

\begin{figure}[]
\captionsetup[subfigure]{}
\centering

\subcaptionbox{(a)}{\includegraphics[width=0.4\columnwidth]{/Users/yehchihhsiang/Desktop/Analysis/CDEX_Analysis_method/Codes/Thesis_Plot/Recoil_Spectrum/0P2GeV_MD_BR_1.pdf}}
\subcaptionbox{(b)}{\includegraphics[width=0.4\columnwidth]{/Users/yehchihhsiang/Desktop/Analysis/CDEX_Analysis_method/Codes/Thesis_Plot/Recoil_Spectrum/0P2GeV_MD_BR_1.pdf}} 
\subcaptionbox{(b)}{\includegraphics[width=0.4\columnwidth]{/Users/yehchihhsiang/Desktop/Analysis/CDEX_Analysis_method/Codes/Thesis_Plot/Recoil_Spectrum/0P2GeV_MD_BR_1.pdf}} 
\subcaptionbox{(b)}{\includegraphics[width=0.4\columnwidth]{/Users/yehchihhsiang/Desktop/Analysis/CDEX_Analysis_method/Codes/Thesis_Plot/Recoil_Spectrum/0P2GeV_MD_BR_1.pdf}} 

\caption{}\label{}
\end{figure}

\begin{figure}[]
\captionsetup[subfigure]{}
\centering

\subcaptionbox{(a)}{\includegraphics[width=0.4\columnwidth]{/Users/yehchihhsiang/Desktop/Analysis/CDEX_Analysis_method/Codes/Thesis_Plot/Recoil_Spectrum/0P2GeV_MD_BR_1.pdf}}
\subcaptionbox{(b)}{\includegraphics[width=0.4\columnwidth]{/Users/yehchihhsiang/Desktop/Analysis/CDEX_Analysis_method/Codes/Thesis_Plot/Recoil_Spectrum/0P2GeV_MD_BR_1.pdf}} 
\subcaptionbox{(b)}{\includegraphics[width=0.4\columnwidth]{/Users/yehchihhsiang/Desktop/Analysis/CDEX_Analysis_method/Codes/Thesis_Plot/Recoil_Spectrum/0P2GeV_MD_BR_1.pdf}} 
\subcaptionbox{(b)}{\includegraphics[width=0.4\columnwidth]{/Users/yehchihhsiang/Desktop/Analysis/CDEX_Analysis_method/Codes/Thesis_Plot/Recoil_Spectrum/0P2GeV_MD_BR_1.pdf}} 

\caption{}\label{}
\end{figure}

\subsection{Sensitivity Line for Quantifying the earth effect}
Since the spectrums from the different channels are already expressed in the previous section, how to quantify the effects for different stages should be addressed. In the following formula, the ratio of survival flux to the original flux, can be adopted to do the quantification:\\
 	\begin{equation}
        \label{Interaction count modification}
	\sigma_{\text{SI}}^{\text{Sensitivity}} = \sigma_{\text{SI}}^{\text{real}}  \times \frac{ \text{Survival Flux}}{\text{Original Flux}}
        \end{equation}
As the cross-section is getting higher, the survival flux will become lower, at certain point, the ratio of that will be zero, which means there is no WIMP that can be detected. There are two examples, which are derived from two experiments, demonstrated at $M_{\chi}$ = 10GeV as follows.\\ 

In figure\ref{}, the sensitivity line for CDEX is shown. The clear pattern is shown that when the cross-section is very small, the flux won't be affected by the earth as the description in stage-1. As the cross-section is getting higher, it alters into the stage-2 , which indicates that the interaction between the atoms of earth and WIMPs initiates. In the end, all of the WIMPs will be hampered by the mountain, which is held accountable for consuming all WIMPs.\\ 

Differing from the underground experiment as CDEX, TEXONO experiment on the surface can accepts more amount of WIMPs having the detector-triggering energy when being under the same cross-section. In Figure\ref{}, the lower $\sigma_{SI}$ that can achieved is expected as there is no other gigantic objects mounted nearby the detector, and all of WIMPs will be impeded by the shielding and building as its transformation into the stage-3.
 
        \begin{figure}[h]
        \includegraphics[scale=0.6]{/Users/yehchihhsiang/Desktop/Analysis/CDEX_Analysis_method/Codes/Thesis_Plot/Sensitivity_Line/KS_Line.pdf}
        \centering
        \caption{} \label{Vacuum-constrained WIMPs}
        \end{figure}

         \begin{figure}[h]
        \includegraphics[scale=0.6]{/Users/yehchihhsiang/Desktop/Analysis/CDEX_Analysis_method/Codes/Thesis_Plot/Sensitivity_Line/CDEX_Line.pdf}
        \centering
        \caption{} \label{Vacuum-constrained WIMPs}
        \end{figure}

\section{Earth-constrained limits on WIMPs}
In the end, earth-constrained limits on WIMPs should be brought out for two experiments. As the sensitivity lines are worked out in the previous sections, the vacuum-constrained limits can be added to the plots for acquiring the corresponding $\sigma_{SI}$ for both upper boundary and lower boundary.\\

In Figure\ref{}, the limit-setting standard plot is expressed with the Migdal effect channel at TEXONO. After the earth-constrained line is completed, which is described in the previous, the  
vacuum-constrained limit can be overlapped together with that. Then, the $\sigma_{SI}$ for the boundaries are crystal clear. 
         \begin{figure}[h]
        \includegraphics[scale=0.6]{/Users/yehchihhsiang/Desktop/Analysis/CDEX_Analysis_method/Codes/Thesis_Plot/Sensitivity_Line/KS_Line_overlap.pdf}
        \centering
        \caption{} \label{Vacuum-constrained WIMPs}
        \end{figure}
        
In finale, the territory on the map of $\sigma_{SI}$ and $M_{\chi}$, which is called "exclusion plot",  based on the capability of exploring the WIMP should be rolled out for the comparisons between different experiments. the bigger the territory that experiment can claim, the higher the possibility of discovering the WIMP they have.\\

As the upper and lower boundaries clarified by the red lines in Figure\ref{} for Migdal effect at $M_{\chi}$=0.8GeV, the boundaries for the rest of the masses can be completed with the same process demonstrated (shown in the appendix) and the enclosed territory can be drafted.\\

To compare with the territories claimed by other experiments working on unveiling WIMP mystery, the territory taken over by our buildup should be overlapped with others in the same plot. Figure\ref{} illustrates the exclusion plot for CDEX with the Migdal effect. In accordance with the sensitivity lines for Brem at CDEX, since the vacuum-constrained $\sigma_{SI}$ is too high to cross the earth-constrained line as an example in Figure\ref{}, leading to the tragedy of losing all of the chances to measure the WIMPs from sub-GeV to GeV mass with Brem channel.

\section{Conclusion}
In conclusion, first the capabilities of our experiments at CDEX and TEXONO on claiming 


